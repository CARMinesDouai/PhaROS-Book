%=================================================================
\ifx%
\wholebook%
\relax%
\else
% --------------------------------------------
% Lulu:
	\documentclass[a4paper,10pt,twoside]{book}
	\usepackage[
		papersize={6.13in,9.21in},
		hmargin={.75in,.75in},
		vmargin={.75in,1in},
		ignoreheadfoot
	]{geometry}
	\input{../common.tex}
	\setboolean{lulu}{true}
% --------------------------------------------
% A4:
%	\documentclass[a4paper,11pt,twoside]{book}
%	\input{../common.tex}
%	\usepackage{a4wide}
% --------------------------------------------
    \graphicspath{{figures/} {../figures/}}
    \begin{document}
\fi
%=================================================================
%\renewcommand{\nnbb}[2]{} % Disable editorial comments
\sloppy
%=================================================================
\chapter{\pharos Cheat sheet}
\chalabel{cheatSheet}
	
	Welcome to PhaROS! \newline
	Remember always to check for updates in \url{http://car.mines-douai.fr/squeaksource/PhaROS.html}. 	 \newline

	\paragraph{Running the generated package \newline}
	
	The generated package has implemented some scripts to show you the basic usage of PhaROS. In PhaROS a package has scripts. Each scripts represents a Node. Each of these nodes share the same class and code. So, take that in count when you implement a new script. 
	
	\paragraph{Basic ROS commands for your generated package \newline}
	
	
	
	For browse your package trough command line. \newline
	
		roscd name-of-your-package \newline 
	
	For editing the image \newline
	
		rosrun name-of-your-package edit  \newline
		
	For running a script with graphical interface  \newline
	
		rosrun name-of-your-package pharos script-name  \newline
	
	 For running a script without graphical interface  \newline
	
		rosrun name-of-your-package headless script-name  \newline
	
	For listing your scripts \newline
	
		rosls name-of-your-package/image/scripts \newline
	
	\section{Inside a package object }
	
	\paragraph{Creating a new Script \newline}


	
		For creating a new script you need to add to your package class a method named script{NameOfYourScript}. Inside this method you will have available a controller, which is an object that gives construction facilities and access to an already built  node. 
		self controller node. 
		
		Inside this method you configure the given node and trigger all the logic of your node. (So, from ROS point of view, each script is a node)
		
		In order to make this script available for execution you have two possibilities: 
		\begin{itemize}
			\item write a text file that executes this method ( PackageName new script{NameOfYourScript}. ) and save it into packageFolder/image/scripts 
			\item commit all your code and use the pharos tool to install it back, this will generate all the needed files. (we are working for making this step easier) 
		\end{itemize}
				
		For examples of what an script is, browse the generated example package and match names with the names available in the script folder. (roscd {package-name}/image/scripts).
		
	\paragraph{Publish topic  \newline}
	
	
	\begin{code}
	| publisher |
	publisher := self controller node 
						topicPublisher: '/example/string' 
						typedAs: 'std-msgs/String'.
						
	publisher send:[ : string | string data: 'this is an example' ].
	
	\end{code}
	
	\paragraph{Subscribe topic \newline}

	
	
	\begin{code}
	(self controller node buildConnectionFor: '/example/string' ) 
			typedAs: 'std-msgs/String'; 
			for: [ : string |  Transcript show: string data ];
			connect .
	\end{code}
	
	\paragraph{Call service \newline}

	\begin{code}
	| service |
	service := self controller node service: '/rosout/get-loggers'.
	service call.
	\end{code }
	
	
	\paragraph{Define service \newline}

	\beign{code}
	
	self controller node serve: [ :req :rsp | 
		 Transcript show: 'Service has been called.'; cr.  
	] at: '/pharos/service' typedAs:'roscpp/Empty'.
	
	\end{code}
	
	
	\paragraph {Inject/install a nodelet \newline}
	
	\begin{code}
	self controller nodelets use: YourNodeletClass as:\#nameToBeInvoked.
	\end{code}
	
	
	\paragraph{Specifying controller configuration \newline}
	
	In the package object implement the message \#buildController. Build controller has the responsibility to build the controller and return it. 
	
	For building your own controller 
	
	
	
	\begin{code}
	buildController
		^ MyController build	
		
	buildController
		\^ self myControllerConfigurationMethod: super buildController.
	
	myControllerConfigurationMethod: aController
		<< Make here your configurations >>
		\^ aController
		
	\end{code}
	
	
	\paragraph{Define a new type \newline}
	
	
	Define as class method a method with a cool name, as myCoolTypeDefinition
	
	
	\begin{code}
	myCoolTypeDefinition
	
	^ PhaROSCompositeType named: 'anStandarROS/TypeName' definedBy: {
		#header -> (PhaROSTypeBrowser instance definition: 'std-msgs/Header'). 
		#auint8 -> (PhaROSUInt8Type new).
		#auint16 -> (PhaROSUInt16Type new).
		#aint32 -> (PhaROSInt32Type new).
		#afloat32 -> (PhaROSFloat32Type new).
		#afloat64 -> (PhaROSFloat64Type new).
		#astring ->( PhaROSStringType new ) .
		#atime -> (PhaROSTimeType new ). 
	}  withConstants: {
		#CONSTANT -> ASimpleObjectValue
	}.
	\end{code}
	As shown in the definition you give an array of associations with (\#nameOfTheField -> Type new). 
	For checking all the available types, just browse any of this classes to go to the package. Or check the reference.
	
	Constants values cannot be complex. Just numbers, strings, booleans. 
	
	
	\paragraph{Register a type \newline}


	Define in class side of your package the method \#types 

	\begin{code}
	types
		^ super types, { #YourTypeName -> self myCoolTypeDefinition }.
	
	\end{code}
	In order to deploy the type into ROS you will need to commit all your work and install it through the pharos command (as shown in the shell commands section). 
	We are working to enhance this step. 
	
	\section{Shell commands}
	
	\paragraph{Install PhaROS based Package\newline}

		pharos install PACKAGE [OPTIONS]\newline
		
		Example \newline
			\-\-pharos install esug --location=/home/user/ros/workspace --version=2.0\newline
		Help 
			\-\-pharos install --help\newline

	\paragraph {Create PhaROS based Package\newline}

		pharos create PACKAGE [OPTIONS]\newline
		
		Example\newline
			\-\-pharos create --location=/home/user/ros/workspace --version=2.0 --author=YourName --author-email=YourEmail \newline
			\-\-Tip: Be sure the email is a correct one. If is not a correctly spelled one you will notice during last step. \newline
		Help \newline
			\-\-pharos create --help\newline
		
		
		
	\paragraph{Register Repository of packages  \newline}
	
		pharos register-repository --url=anUrl --package=aPackage [ OPTIONS ]\newline
		
		Example \newline
			\-\-pharos register-repository --url=http://smalltalkhub.com/mc/user/YourProject/main --package=YourProjectDirectory --directory=YourProjectDirectory \newline
			\-\-Tip: If your repository requires user/password for reading add --user=User --password=Password to the example. \newline
			\-\-Disclaimer: User/Password will be stored in a text file without any security. \newline
		Help 
			\-\-pharos register-repository --help\newline
		
	\paragraph{Listing registered repositories\newline}
	
	
		pharos list-repositories
		
		
	\paragraph{Creating a directory for your own project repository\newline}

		pharos create-repository PACKAGENAME [ OPTIONS ]\newline

		Example \newline
			\-\-pharos create-repository example --user=UserName > directory.st\newline
			\-\-pharos create-repository example --user=UserName  --output= directory.st\newline
		
		Help \newline
			\-\-pharos create-repository --help \newline


%\chapterauthor{\authorjankurs{} \\ \authorguillaume{} \\ \authorlukas{}}


%=============================================================
\ifx\wholebook\relax\else


\end{document}

\fi
%=============================================================
