
%=================================================================
\ifx%
\wholebook%
\relax%
\else
% --------------------------------------------
% Lulu:
	\documentclass[a4paper,10pt,twoside]{book}
	\usepackage[
		papersize={6.13in,9.21in},
		hmargin={.75in,.75in},
		vmargin={.75in,1in},
		ignoreheadfoot
	]{geometry}
	\input{../common.tex}
	\setboolean{lulu}{true}
% --------------------------------------------
% A4:
%	\documentclass[a4paper,11pt,twoside]{book}
%	\input{../common.tex}
%	\usepackage{a4wide}
% --------------------------------------------
    \graphicspath{{figures/} {../figures/}}
    \begin{document}
\fi
%=================================================================
%\renewcommand{\nnbb}[2]{} % Disable editorial comments
\sloppy
%=================================================================
\chapter{Building an image for using PhaROS from scratch}
\chalabel{fromScratch}
\label{appendix:pharos-from-scratch} 
				
						\section{Obtaining a new image}
						
						First we will need a new fresh image. For doing this we will rely on zero conf scripts provided in get.pharo.org
						
						Create a new directory for your installation, step into and execute a zero conf script. By example: 
						
						\begin{lstlisting}[language=bash,title={ Downloading an Pharo 2.0 image }]
							pharos@PhaROS:~$ mkdir myPackage
							pharos@PhaROS:~$ cd myPackage
							pharos@PhaROS:~/myPackage$ wget -O- get.pharo.org/20 | bash
						\end{lstlisting}
						
						
						\begin{lstlisting}[language=bash,title={ Downloading an image - output }]
							--2014-02-21 16:53:39--  http://get.pharo.org/20
							Length: 2587 (2,5K) [text/html]
							Saving to: �STDOUT�
							100%[=====================================>] 2 587       --.-K/s   in 0,004s  
							2014-02-21 16:53:39 (704 KB/s) - written to stdout [2587/2587]
							Downloading the latest 20 Image:
    								http://files.pharo.org/image/20/latest.zip
							Pharo.image
						\end{lstlisting}
						
						Remember that for downloading other versions of pharo you just need to change the last part of the url (20) for the version you want: 14, 20, 30. 
						There is no support PhaROS to run at a Pharo 1.3.
						
						  
						We will use the Pharo 2.0 image for our example. 
						
						
						\begin{lstlisting}[language=bash,title={ Downloading an image - output }]
							pharos@PhaROS:~/myPackage$ pharo-vm-x Pharo
						\end{lstlisting}
					
					
					
							
						\begin{figure}[!htbp]
			  				\centering
    								\includegraphics[width=1\textwidth]{pharos.png}
								\caption{Pharo IDE}
							\centering
						\end{figure}
			
						
						
						\section{Downloading the PhaROS framework}
					
						We will copy then the following code into the workspace 
						
						 
						\begin{code}
							Gofer it url: 'http://car.mines-douai.fr/squeaksource/PhaROS'; 
			     					     package: 'ConfigurationOfPhaROS'; 
			     					     load.
								     
							"If you want to load just the basic stuff, run "
							(Smalltalk at: #ConfigurationOfPhaROS) load: 'default'.
							
							"If you want to load the tests as well, run instead "
							(Smalltalk at: #ConfigurationOfPhaROS) load: 'default+tests'.
						\end{code}
		

						\begin{figure}[!htbp]
			  				\centering
    								\includegraphics[width=1\textwidth]{GoferCodePhaROS.png}
								\caption{Gofer code}
							\centering
						\end{figure}
			
						
						
						
						And execute to load the default package. 
						
						\begin{figure}[!htbp]
			  				\centering
    								\includegraphics[width=1\textwidth]{GoferCodeDoIT.png}
								\caption{Gofer code}
							\centering
						\end{figure}
						
						\begin{figure}[!htbp]
			  				\centering
    								\includegraphics[width=1\textwidth]{GoferCodeLoading.png}
								\caption{Gofer code}
							\centering
						\end{figure}
						
						
						Once all the code is loaded we just need to create our own both pharo and pharos packages. 
						
						
						\section{Making up our package}
						For doing this we need to subclass the class called PhaROSPackage executing the following code
						
						
						\begin{code}
							PhaROSPackage subclass: #[[MyAmazingPhaROSPackage]]
							instanceVariableNames: ''
							classVariableNames: ''
							poolDictionaries: ''
							category: '[[MyCrazyProjectName]]'
						\end{code}
						
						
						Changing, of course, [[MyAmazingPhaROSPackage]] and [[MyCrazyProjectName]] by the names you want.
						
						
						Save the image, You are done. Now you can start to write your scripts as seen before. 
						
						Remember however that automatic script and type generation is not available in this mode.


%\chapterauthor{\authorjankurs{} \\ \authorguillaume{} \\ \authorlukas{}}


%=============================================================
\ifx\wholebook\relax\else


\end{document}

\fi
%=============================================================
